% Options for packages loaded elsewhere
\PassOptionsToPackage{unicode}{hyperref}
\PassOptionsToPackage{hyphens}{url}
%
\documentclass[
]{article}
\usepackage{lmodern}
\usepackage{amssymb,amsmath}
\usepackage{ifxetex,ifluatex}
\ifnum 0\ifxetex 1\fi\ifluatex 1\fi=0 % if pdftex
  \usepackage[T1]{fontenc}
  \usepackage[utf8]{inputenc}
  \usepackage{textcomp} % provide euro and other symbols
\else % if luatex or xetex
  \usepackage{unicode-math}
  \defaultfontfeatures{Scale=MatchLowercase}
  \defaultfontfeatures[\rmfamily]{Ligatures=TeX,Scale=1}
\fi
% Use upquote if available, for straight quotes in verbatim environments
\IfFileExists{upquote.sty}{\usepackage{upquote}}{}
\IfFileExists{microtype.sty}{% use microtype if available
  \usepackage[]{microtype}
  \UseMicrotypeSet[protrusion]{basicmath} % disable protrusion for tt fonts
}{}
\makeatletter
\@ifundefined{KOMAClassName}{% if non-KOMA class
  \IfFileExists{parskip.sty}{%
    \usepackage{parskip}
  }{% else
    \setlength{\parindent}{0pt}
    \setlength{\parskip}{6pt plus 2pt minus 1pt}}
}{% if KOMA class
  \KOMAoptions{parskip=half}}
\makeatother
\usepackage{xcolor}
\IfFileExists{xurl.sty}{\usepackage{xurl}}{} % add URL line breaks if available
\IfFileExists{bookmark.sty}{\usepackage{bookmark}}{\usepackage{hyperref}}
\hypersetup{
  pdftitle={Trend of Current COVID-19 Pandemic in the United States},
  pdfauthor={Jiahe Chen},
  hidelinks,
  pdfcreator={LaTeX via pandoc}}
\urlstyle{same} % disable monospaced font for URLs
\usepackage[margin=1in]{geometry}
\usepackage{graphicx,grffile}
\makeatletter
\def\maxwidth{\ifdim\Gin@nat@width>\linewidth\linewidth\else\Gin@nat@width\fi}
\def\maxheight{\ifdim\Gin@nat@height>\textheight\textheight\else\Gin@nat@height\fi}
\makeatother
% Scale images if necessary, so that they will not overflow the page
% margins by default, and it is still possible to overwrite the defaults
% using explicit options in \includegraphics[width, height, ...]{}
\setkeys{Gin}{width=\maxwidth,height=\maxheight,keepaspectratio}
% Set default figure placement to htbp
\makeatletter
\def\fps@figure{htbp}
\makeatother
\setlength{\emergencystretch}{3em} % prevent overfull lines
\providecommand{\tightlist}{%
  \setlength{\itemsep}{0pt}\setlength{\parskip}{0pt}}
\setcounter{secnumdepth}{-\maxdimen} % remove section numbering

\title{Trend of Current COVID-19 Pandemic in the United States}
\author{Jiahe Chen}
\date{}

\begin{document}
\maketitle

\hypertarget{introduction}{%
\subsection{Introduction}\label{introduction}}

The objectives of this project is to explore 1) whether the COVID-19
pandemic is ending (i.e., are the numbers of new cases/deaths
decreasing) specifically in the United States and 2) whether the current
lockdown policies are effective against COVID-19 transmission (i.e.,
whether there are significant differences between the numbers of new
cases/deaths pre- versus post-lockdown.

The data being used for the project is the daily data of cumulative
confirmed cases of COVID-19 and cumulative deaths from COVID-19 in the
United States, by county and since Jan 22nd, 2020. It is released and
currently being updated on daily basis by Johns Hopkins University
Center for Systems Science and Engineering.

\hypertarget{methods}{%
\subsection{Methods}\label{methods}}

The original dataset can be found in JHU CSSE official github repository
for COVID-19 data: \url{https://github.com/CSSEGISandData/COVID-19}.
There are 3338 rows and 311 columns in the raw data of cases, 3338 rows
and 312 columns in the raw data of deaths.

First, the unneeded variables (for example, ``Country'' is useless in
this project because all the interested data is for the United States)
is dropped from the dataset. The original dataset also includes cases
and deaths in places outside the United States (e.g., the diamond
princess cruise ship); as the populations on cruise ships are poorly
defined and are relatively small, they are dropped from the dataset. The
data for American territories are also dropped for similar reasons.
Missing values are checked and no missing value is observed for daily
reported cases and deaths.

The case and death datasets are then 1) reshaped from long to wide and
2) combined for easier interpretation. Variables are renamed
accordingly. The county level numbers within each state are added up to
get the state level data, and state level numbers are added up to get
the US national level data, listed as ``US total''. The cumulative
case/death of the previous date is subtracted from that of the current
date to calculate the daily new case/death. Because of that, the data of
very first recorded date (Jan 22th) is invalid and dropped. The log
transformation is used for better visualization in figures; when the log
transformation produced -Inf due to 0s in the raw data, the -Infs are
replaced by 0s for easier processing.

Specially, the JHU COVID-19 datasets do not include the abbreviations of
US states. As such, a third-party dataset which converts full names of
US states to their standard abbreviations was found in Github and used
in this project:
\url{https://github.com/jasonong/List-of-US-States/raw/master/states.csv}.
Finally, the dataset is sorted according to state name and date.

\hypertarget{table-1-descriptive-stats-of-covid-19-daily-new-casesdeaths-since-jan-23rd}{%
\subsection{Table 1: Descriptive Stats of COVID-19 Daily New
Cases/Deaths Since Jan
23rd}\label{table-1-descriptive-stats-of-covid-19-daily-new-casesdeaths-since-jan-23rd}}

\hypertarget{htmlwidget-d77711810ca43d831642}{}
\begin{datatables}

\end{datatables}

\hypertarget{figure-1-time-series-of-log-of-covid-19-daily-new-casesdeaths-in-california-and-the-united-states}{%
\subsection{Figure 1: Time series of Log of COVID-19 Daily New
Cases/Deaths in California and the United
States}\label{figure-1-time-series-of-log-of-covid-19-daily-new-casesdeaths-in-california-and-the-united-states}}

\hypertarget{cases}{%
\paragraph{Cases}\label{cases}}

\hypertarget{htmlwidget-e62dd78dc574b438acee}{}
\begin{plotly}

\end{plotly}

\hypertarget{deaths}{%
\paragraph{Deaths}\label{deaths}}

\hypertarget{htmlwidget-87bb61489e186ea2cd9f}{}
\begin{plotly}

\end{plotly}

\hypertarget{figure-2-geo-plot-of-differences-of-daily-new-casesdeaths-nov-16-vs.-may-1-by-state}{%
\subsection{Figure 2: Geo plot of Differences of Daily New Cases/Deaths
Nov 16 vs.~May 1 by
State}\label{figure-2-geo-plot-of-differences-of-daily-new-casesdeaths-nov-16-vs.-may-1-by-state}}

\hypertarget{cases-1}{%
\paragraph{Cases}\label{cases-1}}

\hypertarget{htmlwidget-857da3f1d3769fb5d963}{}
\begin{plotly}

\end{plotly}

\hypertarget{deaths-1}{%
\paragraph{Deaths}\label{deaths-1}}

\hypertarget{htmlwidget-2a044e74b233fbb7bcbc}{}
\begin{plotly}

\end{plotly}

\hypertarget{conclusion}{%
\subsection{Conclusion}\label{conclusion}}

Based on the table and figure, we conclude that the pandemic is still
far from ending in the United States in terms of cases, but is getting
better in terms of deaths. Although the recent daily new cases have
dropped significantly from the July and August peak, today's new cases
are still quite high. The time series figure suggests that the daily new
cases do not have a clear declining trend yet, and the pre-post
comparison in geoplot figure suggests that the current lockdown policies
are not quite effective against this pandemic so far.

\end{document}
